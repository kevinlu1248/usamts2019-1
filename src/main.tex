%! Author = kevin
%! Date = 2019-09-10

% Preamble
\documentclass{article}
\usepackage[utf8]{inputenc}
\usepackage[english]{babel}

% Packages
\usepackage{amsmath}
\usepackage{amsthm}
%\usepackage{gensymb}

% Custom functions
\renewcommand{\thesection}{Section \arabic{section}\\ }
\newtheorem{lemma}{Lemma}

\title{USAMTS 2019 -2020 Set 1}
\author{Kevin Lu}
\date{September 2019}

% Document
\begin{document}

    \maketitle

    \section{}\label{sec:}

    \section{}\label{sec:2}
    \begin{lemma}[Particular Case of the Rearrangement Inequality]
        If $a_1\geq a_2 \geq a_3$ and $b_1\leq b_2 \leq b_3$ then $a_1b_1+a_2b_2+a_3b_3\leq a_{1b}_2+a_2b_3+a_3b_1$ with equality iff ($a_1=a_2$ or $b_1=b_2$) and ($a_1=a_3$ or $b_2=b_3$).
    \end{lemma}
    \begin{proof}
        The inequality case is trivialized by the rearrangement inequality but the equality case may require a bit more work. Since $(a_1-a_2)(b_1-b_2)\leq0 \implies a_1b_1+a_2b_2+a_3b_3\leq a_1b_2+a_2b_1+a_3b_3$ with equality when $a_1=a_2$ or $b_1=b_2$. Since $(a_1-a_3)(b_3-b_2)\geq0 \implies a_1b_2+a_2b_1+a_3b_3\leq a_1b_3+a_2b_1+a_3b_2$ with equality when $a_1=a_3$ or $b_2=b_3$. Thus, we have $a_1b_1+a_2b_2+a_3b_3\leq a_1b_2+a_2b_1+a_3b_3\leq a_1b_3+a_2b_1+a_3b_2$ with equality iff ($a_1=a_2$ or $b_1=b_2$) and ($a_1=a_3$ or $b_2=b_3$).
    \end{proof}
    Rearranging the given yields the following
    \begin{align*}
        \sqrt[y]{z}&=\sqrt[x]{x} \\
        \sqrt[z]{x}&=\sqrt[y]{y} \\
        \sqrt[x]{y}&=\sqrt[z]{z} \\
    \end{align*}
    Multiplying the three yields
    \begin{equation}
        \sqrt[z]{x}\sqrt[x]{y}\sqrt[y]{z}&=\sqrt[x]{x}\sqrt[y]{y}\sqrt[z]{z}
    \end{equation}
    WLOG let $x\geq y\geq z$. Lemma 1 with $\log{x}\geq\log{y}\geq\log{z}$ and $\frac{1}{x}\leq\frac{1}{y}\leq\frac{1}{z}$ yields
    \begin{align*}
        \frac{1}{x}\log{x}+\frac{1}{y}\log{y}+\frac{1}{z}\log{z}&\leq \frac{1}{z}\log{x}+\frac{1}{x}\log{y}+\frac{1}{y}\log{z}\\
        log{(\sqrt[z]{x}\sqrt[x]{y}\sqrt[y]{z})}&\leq\log{(\sqrt[x]{x}\sqrt[y]{y}\sqrt[z]{z})} \\
        \sqrt[z]{x}\sqrt[x]{y}\sqrt[y]{z}&\leq\sqrt[x]{x}\sqrt[y]{y}\sqrt[z]{z}
    \end{align*}
    with equality iff $\frac{1}{x}=\frac{1}{y}$ or $\log{x}=\log{y}$ (ignoring the second requirement of the "and" since the first is sufficient), either of which implies $x=y$. Since $x^y=z^x$ and $x,y,z\geq1$, this implies $x^x=z^x\implies x=z\implies x=y=z$, as required.

    \section{}\label{sec:3}
    Let $[X_1X_2\cdots X_n]$ be the area of n-gon $X_1X_2\cdots X_n$. Let O be the center of $\omega$ and X and Y the respective projections of O and P on line $\overline{IE}$. Since $[EOI]=\frac{EI\cdot OX}{2}$, $[EPI]=\frac{EI\cdot PY}{2}$ and $UOE\sim UPY$ by AA with angles $\angle OXU=90^\circ =\angle UPY$ and $\angle OUE=\angle PUY$,
    \begin{equation}
        \frac{[EOI]}{[EPI]}=\frac{OX}{PY}=\frac{UO}{UP}=\frac{1}{2} \implies [EOI]=\frac{[EPI]}{2}.
    \end{equation}
    Let $\theta=\angle EOU$ and $r=1/2$ be the radius of $\omega$. We have
    \begin{equation}
        [EOI]=\frac{EI\cdot OX}{2}=EX\cdot OX=r\sin{\theta}\cdot r\cos{\theta}=\frac{r^2\sin{2\theta}}{2}=\frac{\sin{2\theta}}{8}
    \end{equation}
    Since $\sin{2\theta}\leq 1$ with equality when $2\theta=90^\circ\implies \theta=45^\circ$, we have $[EPI]=2[EOI]\leq \frac{1}{4}$, with equality when $\angle EOI=45^\circ$.
    \section{}\label{sec:4}

    \section{}\label{sec:5}


\end{document}